\documentclass[russian,utf8,floatsection,equationsection]{eskdtext}
% Объявляем документ класса eskdtext (подробнее можно узнать из описания пакета eskdx)
% russian - текст на русском языке, utf8 - кодировка документа UTF-8
% floatsection - нумерация таблиц и рисунков с учётом номера главы, equationsection - то же для формул
\usepackage{longtable} % В документе используем пакет longtable для создания таблиц
\usepackage{graphicx} % Используем графику в документе
\usepackage{tikz}
\usepackage{mathtext} % Русские буквы в формулах
\usepackage[T2A]{fontenc}

\usetikzlibrary{shapes,arrows}
 
\ESKDdocName{Проектирования системы управления на базе обучения с подкреплением}
\ESKDsignature{Курсовой проект по проектированию}
\ESKDsignature{КуП 15.04.06\_03 СМ7} % Шифр
 
 \ESKDdepartment{%
 Московский Государственный Технический Университет имени Н.Э.Баумана \\
(Национальный Исследовательский Университет) 
}
\ESKDcompany{%
 \includegraphics[width=0.3\textwidth]{img/logo.png}\\[1cm]
}
\ESKDauthor{Юнес А.Ю.} % "Разраб." в штампе на листе содержания
\ESKDchecker{Ющенко А.С.} % "Пров."  в штампе на листе содержания
% \ESKDnormContr{} % "Н. контр." в штампе на листе содержания
% \ESKDapprovedBy{}%  "Увт." в штампе на листе содержания
% \ESKDdate{2019} % Дата (Год отображается на титульной странице)
% \ESKDletter{}{У}{} % Литеры
 
\renewcommand{\ESKDtheTitleFieldX}{%
Москва
\ESKDtheYear~г.} % Шаблон для отображения в нижней части титульного листа города и года

\ESKDtitleDesignedBy{Студент}{Юнес \textsc{А.Ю.}} % Подпись и дата под заголовком документа
\ESKDtitleApprovedBy{Рукаводитель:}{Ющенко \textsc{А.С.}} % Утверждаю
 
\renewcommand{\baselinestretch}{1} % Задаём единичный межстрочный интервал
 
\ESKDsectStyle{section}{\normalsize} % Заголовки глав обычным шрифтом
%\ESKDsectStyle{subsection}{\normalsize} % Заголовки разделов обычным шрифтом
%\ESKDsectStyle{subsubsection}{\normalsize} % Заголовки подразделов обычным шрифтом
 
\begin{document} % Маркер начала документа
\maketitle % Создать титульный лист на основе данных в заголовке документа

\tikzstyle{rect}=[draw,rectangle,fill=white!20,text width=6em,text centered,minimum height=2em]
\tikzstyle{elli}=[draw,ellipse,fill=white!20]%,minimum height=2em]
\tikzstyle{circ}=[draw,circle,fill=white!20]%,minimum width=8pt,inner sep=10pt]
\tikzstyle{diam}=[draw,diamond,fill=white!20,text width=6em,text badly centered,inner sep=0pt,, aspect=1.5]
\tikzstyle{line}=[draw,-latex']
\newpage
\ESKDappendix{Algorithm}{MPC controller\label{app:tt120}}
\begin{figure}[h]
    \centering
    % node distance=3cm
    % don't forget ;
    \begin{tikzpicture}[node distance=1.5cm,auto]
    \node [elli](a1){start};
    \node [rect, below of=a1, rounded corners](a2){calculate};
    \node [diam,below of=a2,node distance=3cm](d1){OK? or you have another opinion};
    \node [rect,left of=d1, node distance=5cm](a3){aa};
    \node [rect,below of=d1, node distance=3cm](a4){aasa};
    \path [line] (a1) -- (a2);
    \path [line] (a2) -- (d1);
    \path [line] (d1) -- node [above, near start] {No} (a3);
    \path [line] (d1) -- node [left, near start] {yes} (a4);
    \path [line, rounded corners,dashed] (a3) |- (a2); 
    \end{tikzpicture}
    \caption{Caption}
    \label{fig:my_label}
\end{figure}
\newpage
\ESKDappendix{Algorithm}{CEM optimizer\label{app:tt120}}
\newpage
\ESKDappendix{Algorithm}{CEM optimizer\label{app:tt120}}
\newpage
\ESKDappendix{Algorithm}{CEM optimizer\label{app:tt120}}

\end{document} % Маркер завершения документа